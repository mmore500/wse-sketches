\section{Key Considerations} \label{sec:key-Considerations}

\begin{enumerate}
    \item \textit{Using the SDK requires programming in CSL, a C-like language designed specifically for the problem of massively parallel programming on the CS-2.
    What is your experience with HPC programming paradigms and languages, such as MPI, OpenMP, CUDA, OpenCL, etc.?}
    \item \textit{What are the underlying computational algorithms you're interested in exploring?
    What existing software packages or libraries use these algorithms?}
    \item \textit{How is this problem bottlenecked on current hardware?
    Is the problem more bottlenecked by memory bandwidth, or communication costs associated with scaling up across distributed compute nodes?}
    \item \textit{What range of problem sizes are you interested in addressing?
    For example, how much memory does your problem use?
    How does memory usage or program run-time scale with problem size?}
    \item \textit{What portion of your algorithm do you plan to port to the CS-2?
    Why are you interested in exploring this part of your algorithm?}
    \item \textit{The CS-2 offers native half and single precision data types.
    What precision does your algorithm or use case need?}
    \item \textit{The CS-2 is a network-attached accelerator. At a high level, the CSL programming model is similar to that of CUDA, in which data is moved between host CPU nodes and the device (CS-2) on which computational kernels are launched.}
    How often will data need to be moved between the wafer and the worker nodes?
    \item \textit{Describe your general plan to map your problem onto 850,000 cores. In answering this question, it might be helpful to recall some details of the CS-2 architecture.
    The 850,000 cores are laid out in a mesh, with each core connected on fabric to its four nearest neighbors on the East, South, North, and West.
    Memory is distributed among the cores, with each core having 48 KB of local memory.
    64-bit local memory reads and writes take roughly a cycle, as does sending and receiving 32-bit messages between the four neighboring cores.}
\end{enumerate}
