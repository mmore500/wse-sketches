\section{Key Considerations} \label{sec:key-considerations}

\begin{enumerate} \itshape \bfseries
\item %
% Using the SDK requires programming in CSL, a C-like language designed specifically for the problem of massively parallel programming on the CS-2.
What is your experience with HPC programming paradigms and languages, such as MPI, OpenMP, CUDA, OpenCL, etc.?
\end{enumerate}
\noindent
Dr. Moreno has extensive experience working with HPC technologies, most notably developing asynchronous, best-effort computing approaches using MPI and C++.
He has also received training in GPU programming (e.g., OpenCL, CUDA).
Dr. Dolson and Dr. Zaman are both experienced systems-level programmers.

\begin{enumerate}[resume] \itshape \bfseries
\item What are the underlying computational algorithms you're interested in exploring?
What existing software packages or libraries use these algorithms?
\end{enumerate}
\noindent
The core algorithm is an island-style GA, which is well-supported by digital evolution frameworks targeting serial processing on traditional HPC hardware \citep{bohm2017mabe}.
Those that support parallelism generally apply a centralized coordinator-worker strategy \citep{fortin2012deap}, which is best suited to small population sizes with expensive fitness evaluations.

\begin{enumerate}[resume] \itshape \bfseries
\item How is this problem bottlenecked on current hardware?
% Is the problem more bottlenecked by memory bandwidth, or communication costs associated with scaling up across distributed compute nodes?
\end{enumerate}
\noindent
Memory bandwidth and synchronization costs to exchange population members.
Compute bound memory latency.

\begin{enumerate}[resume] \itshape \bfseries
\item What range of problem sizes are you interested in addressing?
% For example, how much memory does your problem use?
How does memory usage or program run-time scale with problem size?
\end{enumerate}
\noindent
We are interested in developing population-scale, cross-scale simulations with hundreds of millions of agents.
The memory usage of the program and run-time scale linearly with problem size, either defined as evolutionary duration or population size.
Precise memory usage depends largely on the agent genomes used.
For smaller agent genomes, sized on the order of four words, thousands of agents could be supported per PE --- yielding a net whole-wafer population size on the order of a billion agents.

\begin{enumerate}[resume] \itshape \bfseries
\item What portion of your algorithm do you plan to port to the CS-2?
% Why are you interested in exploring this part of your algorithm?
\end{enumerate}
\noindent
We plan to port the core evolutionary generational loop to the CS-2 as an island model-style genetic algorithm.
The stages are 1) migration, 2) selection, 3) replication with mutation.
We are interested in porting it because it is the most computationally intensive to scale and because it is embarrassingly parallel.
Postprocessing to analyze evolved traits, and reconstruct evolutionary history will take place off-device.

\begin{enumerate}[resume] \itshape \bfseries
\item %
% The CS-2 offers native half and single precision data types.
What precision does your algorithm or use case need?
\end{enumerate}
\noindent
Due to the agent-based modeling, this is a primarily discrete domain.
However, floating point types may be used to represent traits.
Because they are only used for comparisons between agents, single or half precision suffice.

\begin{enumerate}[resume] \itshape \bfseries
\item %
% The CS-2 is a network-attached accelerator. 
% At a high level, the CSL programming model is similar to that of CUDA, in which data is moved between host CPU nodes and the device (CS-2) on which computational kernels are launched.
How often will data need to be moved between the wafer and the worker nodes?
\end{enumerate}
\noindent
For preliminary versions of the project, we are planning to only collect data at the end of execution through the Cerebras memcpy infrastructure.
As the model is self-contained, no data except for the PE code will need to be moved onto the device.
In the future, we are interested in exploring samples throughout the evolutionary process to serve as a kind of ``fossil record.''

\begin{enumerate}[resume] \itshape \bfseries
\item Describe your general plan to map your problem onto 850,000 cores. 
% In answering this question, it might be helpful to recall some details of the CS-2 architecture.
% The 850,000 cores are laid out in a mesh, with each core connected on fabric to its four nearest neighbors on the East, South, North, and West.
% Memory is distributed among the cores, with each core having 48 KB of local memory.
% 64-bit local memory reads and writes take roughly a cycle, as does sending and receiving 32-bit messages between the four neighboring cores.
\end{enumerate}
\noindent
We plan to exploit an island model genetic algorithm where each PE is an island and genomes migrate between adjacent islands.
As natural systems are typically asynchronous, synchronization is not a concern.
We plan to use CSL asynchronous send/receive operations to implement migration.

Genomes will be represented as contiguous buffers of u32 and each PE's population as a contiguous buffer of genome data using a simple synchronous generational model (i.e., a swap buffer.)
Population-level operations (e.g., selection, reproduction, mutation) will take advantage of the CSL's data structure descriptors.
The local memory is sufficient for a population size on the order of hundreds or low thousands.
This could give a global population size on the order of hundreds of millions.
