\section{Conclusion} \label{sec:conclusion}

In this proposal, we lay foundations to harness wafer-scale computing to achieve order-of-magnitude scale-up for agent-based modeling.
This will allow researchers to study entirely new classes of cross-scale dynamics currently out of reach in many ABM application areas.
To advance on this front, we explore a fundamental re-frame of simulation that shifts from a paradigm of ``complete,'' perfect data observability to dynamic, approximate observability akin to estimation approaches traditionally used to study physical systems.

Within ABM, our proposed work focuses in particular on the topic of evolutionary computation.
We will harness new cutting-edge approaches to distributed phylogenetic tracking to answer fundamental questions about the relationship between phylogenetic structure and population scale.
These new methods, and extensive development targeting the Cerebras CSL using the SDK hardware simulator, position our project to hit the ground running.
In conducting experiments evaluating phylogenetic structure, we include steps to demonstrate and evaluate the underlying introduced novel technical approaches.

Potential extensions of this work repurposing dynamic cross-temporal sampling algorithms designed for HStrat to target new applications BMS/PDES domains and increase their observability by dnyamically recording time series activity at critical simulation sites or to extract coarsened histories of individual agents like position across time.
Within our own research agenda, this work will scaffold more targeted work in evolutionary epidemiology We will use our expanded scale to capture within-host and between-host pathogen behaviors, and evaluate evolved pathogen traits that balance trade-offs occurring across scales.
We hope our findings, and reusable software tools produced for the work, catalyze transformative research in this area.
